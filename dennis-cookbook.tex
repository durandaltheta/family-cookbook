\documentclass{article}
\usepackage{framed}
\usepackage{fancyhdr}
\usepackage{lipsum}% just to generate text for the example
\pagestyle{fancy}
\fancyhf{}
\fancyhead[C]{\leftmark}
\fancyhead[R]{\thepage}
\renewcommand{\headrulewidth}{0pt}

\title{Dennis Family Cookbook}
\date{}
\begin{document}
\maketitle
\tableofcontents
\newpage

% =====================================================
\vspace*{\fill}
\begin{center}
    \section{Breakfasts}
\end{center}
\vspace*{\fill}
\newpage 

% -----------------------------------------------------
\subsection{Cowboy Coffee Cake} 
\noindent\rule[0.5ex]{\linewidth}{1pt}

%Ingredients
\begin{framed}
    \begin{itemize}
        \item 1 tbsp vinegar 
        \item ~1 cup milk
        \item 2 1/2 cups flour
        \item 2 cups brown sugar 
        \item 1/2 teaspoon salt 
        \item 2/3 cups shortening
        \item 2 tsp baking powder
        \item 1/2 tsp baking soda 
        \item 1/2 tsp cinnamon  
        \item 1/2 tsp nutmeg 
        \item 2 well beaten Eggs
    \end{itemize}
\end{framed}

%Instructions
\begin{enumerate}
    \item 
        Create sour milk by putting vinegar in an empty 1 cup measuring cup. Fill remaining space in the cup with with milk.
    \item 
        Combine flour, brown sugar, salt, and shortening till crumbly. Reserve 1/2 cup of mixture to crumble over batter later.
    \item 
        Add baking power, baking soda, cinnamon and nutmeg remaining batter. Mix thoroughly.
    \item 
        Add sour milk and well beaten eggs. Mix well.
    \item 
        Line 2 8x8 square pans with wax paper. Pour batter into pans. Sprinkle with reserved crumbs.
    \item 
        Bake at 375 degrees for approx. 25 minutes.
\end{enumerate}
\newpage

% =====================================================
\vspace*{\fill}
\begin{center}
    \section{Soups}
\end{center}
\vspace*{\fill}
\newpage

% -----------------------------------------------------
\subsection{Chili con Carne} 

%Ingredients
\begin{framed}
    \begin{itemize}
        \item 1 lb ground beef (or turkey)
        \item 1 C chopped onion
        \item 3/4 C green pepper
        \item 1 clove garlic, minced
        \item 1 28 oz can tomatoes, cut up
        \item 2 16 oz cans dark red kidney beans, drained
        \item 1 16 oz can tomato sauce
        \item 3 oz (1/2 can) tomato paste
        \item 1 T chili powder
        \item 1 tsp dried basil, crushed
        \item 1/2 tsp salt
        \item 1/4 tsp pepper
    \end{itemize}
\end{framed}

%Instructions
\begin{enumerate}
    \item
        In a large kettle cook ground beef, onion, green pepper, and garlic until meat is browned. Drain off fat.
    \item
        Stir in undrained tomatoes, kidney beans, tomato sauce, chili powder, basil, salt, and pepper. Bring to boiling; reduce heat.
    \item
        Cover and simmer about 20 minutes
    \item
        Makes 4-6 servings
\end{enumerate}
\newpage

% -----------------------------------------------------
\subsection{Caldo Verde} 
\noindent\rule[0.5ex]{\linewidth}{1pt}

%Ingredients
\begin{framed}
\begin{itemize}
    \item 1/4 cup olive oil
    \item 1 cup chopped onion
    \item 2 tsp chopped garlic
    \item 5 cups Ihaho potatoes, peeled and thinly sliced
    \item 1 quart water
    \item 1 quart chicken broth
    \item 6 oz chorizo sausage, thinly sliced
    \item salt and black pepper
    \item 1 lb kale, washed, trimmed of the thick stems and thinly sliced
\end{itemize}
\end{framed}

%Instructions
\begin{enumerate}
\item 
    In a medium soup pot, heat 3 tablespoons of olive oil, add onions and garlic and cook for 2 to 3 minutes until they turn glassy, don't let them get brown. 
\item 
    Add potatoes and water. Cover and boil gently over medium heat for 20 minutes. Meanwhile, in a skillet cook sausage until most of the fat is rendered out. Drain and set aside. 
\item 
    When the potatoes are tender, mash them with a potato masher right in the pot. Add sausage to the soup then add kale. Simmer for 5 minutes. Add the remaining olive oil and season. Ladle into bowls and serve.
\end{enumerate}
\newpage

% -----------------------------------------------------
\subsection{Fruit Soup} 
\noindent\rule[0.5ex]{\linewidth}{1pt}

%Ingredients
\begin{framed}
\begin{itemize}
    \item 1 cup raisins 
    \item 1 cup prunes 
    \item 1/2 cup rice or tapioca 
    \item 1 orange 
    \item 1 lemon 
    \item 1 cup sugar 
    \item 2 cinnamon sticks 
    \item 2 apples
    \item Any fruit juice cocktail 
    \item 2 quarts water 
\end{itemize}
\end{framed}

%Instructions
\begin{enumerate}
\item 
    Peel and dice apples. Juice orange and lemon.
\item 
    Place all ingredients in a large pot. Simmer slowly for several hours stirring occasionally until thickened.
\end{enumerate}
\newpage

% =====================================================
\vspace*{\fill}
\begin{center}
    \section{Breads}
\end{center}
\vspace*{\fill}
\newpage

% -----------------------------------------------------
\subsection{Pioneer Cornbread} 
\noindent\rule[0.5ex]{\linewidth}{1pt}

%Ingredients
\begin{framed}
    \begin{itemize}
        \item 1 1/3 C Cold Margerine or Butter (frozen is best)
        \item 1 C sugar
        \item 2 Eggs
        \item 1 C cornmeal
        \item 1 tsp salt
        \item 2 C flour
        \item 1 T baking powder
        \item 2 C milk
    \end{itemize}
\end{framed}


%Instructions
\begin{enumerate}
    \item 
        Cut margerine/butter into small cubes.
    \item 
        Mix margerine/butter with sugar and eggs. Mix well and make sure the cubes are well separated and not clumped together
    \item 
        Add remaining ingredients and stir until moistened
    \item 
        Pour into a greased 9x13 pan or two greased cast iron skillets
    \item 
        Bake at 350 degrees for 20-30 min. Cornbread is done when a toothpick inserted in the middle comes out clean
    \item 
        If desired: remove pan from the oven and turn on the broiler, move the rack to the top position and brown the cornbread under the boiler for 30 seconds or so.  Watch carefully the entire time.  The bread can turn from brown to burned in seconds!
\end{enumerate}
\newpage

% =====================================================
\vspace*{\fill}
\begin{center}
    \section{Appetizers \& Side Dishes}
\end{center}
\vspace*{\fill}
\newpage

% -----------------------------------------------------
\subsection{Cranberry Cherry Jello Salad} 
\noindent\rule[0.5ex]{\linewidth}{1pt}

%Ingredients
\begin{framed}
    \begin{itemize}
        \item 2 cups fresh cranberrys
        \item 1 cup sugar
        \item 1 1/2 cup water
        \item 3 oz cherry jello 
        \item 3/4 cup chopped celery 
        \item 1/2 cup chopped nuts 
        \item 1/2 cup chopped apples
    \end{itemize}
\end{framed}

%Instructions
\begin{enumerate}
    \item 
        Boil berries, sugar, and water until berries pop, remove from heat.
    \item 
        Add jello and let cool slightly.
    \item 
        Mix in celery, nuts and apples. Put in fridge to chill for a couple hours.
\end{enumerate}
\newpage

% -----------------------------------------------------
\subsection{Red Hot Jello} 
\noindent\rule[0.5ex]{\linewidth}{1pt}

%Ingredients
\begin{framed}
    \begin{itemize}
        \item 1/2 cup water
        \item 1 pkg red jello 
        \item 1/4 cup red hot candies (generally cinnamon candies)
        \item 1 applesauce
    \end{itemize}
\end{framed}

%Instructions
\begin{enumerate}
    \item 
        Boile water and add water to red jello and candies. Dissolve candies.
    \item 
        Mix in applesauce. Chill in fridge for a couple hours.
\end{enumerate}
\newpage

% =====================================================
\vspace*{\fill}
\begin{center}
    \section{Vegetables}
\end{center}
\vspace*{\fill}
\newpage

% =====================================================
\vspace*{\fill}
\begin{center}
    \section{Entrees}
\end{center}
\vspace*{\fill}
\newpage

% =====================================================
\vspace*{\fill}
\begin{center}
    \section{Deserts}
\end{center}
\vspace*{\fill}
\newpage

% -----------------------------------------------------
\subsection{Crock Pot Candy} 
\noindent\rule[0.5ex]{\linewidth}{1pt}

%Ingredients
\begin{framed}
\begin{itemize}
    \item 32 oz. mixed nuts with extra cashews 
    \item 12 oz. semi sweet chocolate chips 
    \item 1 bakers 4 ozs. german chocolate bar
    \item 32 oz white chocolate chips
\end{itemize}
\end{framed}

%Instructions
\begin{enumerate}
\item 
    Put mixed nuts in crockpot. 
\item 
    Pour chocolate chips over nuts.
\item 
    Break german chocolate bar over top.
\item 
    Pour white chocolate chips over everything.
\item 
    Cook on low (DO NOT REMOVE LID) for 2 hours.
\item 
    Stir well and drop on waxed paper to cool.
\end{enumerate}
\newpage

% -----------------------------------------------------
\subsection{Mystery Desert} 
\noindent\rule[0.5ex]{\linewidth}{1pt}

%Ingredients
\begin{framed}
\begin{itemize}
    \item 2 cup flower 
    \item 1 cup Butter
    \item 1 cup chopped walnuts
    \item 9 oz cool whip
    \item 8 oz cream cream cheese 
    \item 2/3 cup powdered sugar
    \item 2 packages chocolate instant Pudding
    \item cool whip (as a topping)
\end{itemize}
\end{framed}

%Instructions
\begin{enumerate}
\item 
    Bake flour, butter, and walnuts at 350 degrees 15minutes on 9x13 inch pan. Let cool
\item 
    Mix 9 oz cool whip, cream cheese and powdered sugar together. Layer mixtur on top of bottom crumble.
\item 
    Make chocolate pudding according to instructions on package. Layer pudding on top of previous layer in pan.
\item 
    Layer cool whip as final layer on pan. 
\end{enumerate}
\newpage

% -----------------------------------------------------
\subsection{Rice Pudding} 
\noindent\rule[0.5ex]{\linewidth}{1pt}

%Ingredients
\begin{framed}
    \begin{itemize}
        \item 1 cup white rice.
        \item approx. 2 cups water
        \item 1 quart milk 
        \item 1 can evaporated milk 
        \item 1 cup sugar 
        \item 3 beaten eggs 
        \item 1 tsp vanilla extract
        \item Raisins to taste
        \item pinch of cinnamon
    \end{itemize}
\end{framed}

%Instructions
\begin{enumerate}
    \item 
        Cook rice according to packaging directions.
    \item 
        Add milk and evaporated milk to rice in a pot. Bring to a boil stirring often. Add sugar, stir, and boil *slowly* for about 20 minutes stirring constantly. 
    \item 
        Beat eggs in a separate bowl and slowly add a tbsp or two of the rice mixture at a time to the eggs, stirring well after each addition, until you've added about 1 cup of rice mixture to the eggs and both are mixed together well. 
    \item 
        Mix egg/rice mixture into the original pot with the rest of the heated rice mixture. Boil an additional 4 or 5 minutes stirring slowly.
    \item 
        Remove form heat, add vanilla extract and raisins. Mix well. Sprinkle cinnamon on top. Chill for a few hours till quite cool.
\end{enumerate}
\newpage

% -----------------------------------------------------
\subsection{Tom Dennis Master Custard Ice Cream Recipe} 
\noindent\rule[0.5ex]{\linewidth}{1pt}

%Ingredients
\begin{framed}
    \begin{itemize}
        \item 2 cups whole milk
        \item 1 cup sugar
        \item 4 egg yolks
        \item pinch of salt
        \item 2 cups of 1/2\&1/2 (milk and cream base, non-alcoholic)
        \item 2 cups Cream
        \item 3 teaspoons vanilla extract
    \end{itemize}
\end{framed}

%Instructions
\begin{enumerate}
    \item 
        In pan wisk milk, sugar, egg yokes, and salt on medium heat until mixture simmers.
    \item 
        Lower heat, wisk 5 minutes till mixture thickens.
    \item 
        Strain into a bowl and wisk in 1/2\&1/2, cream, and vanilla.
    \item 
        Churn in an ice cream machine according to manufacturer's instructions. Serve directly from machine for soft serve, or store in freezer till needed
\end{enumerate}

%Flavors
\noindent\rule[0.5ex]{\linewidth}{0.5pt}
\paragraph 
(The following flavors can create thicker custard than usual, be aware that manual churning may be required if the ice cream machine is not powerful enough. Alternatively, one can flip the ratio of milk to cream base to the following alternate measurements to thin the recipe as provided in each flavor.)

\subsubsection{Almond Flavor}
\begin{framed}
    \begin{itemize}
        \item alternate: 3 cups whole milk
        \item alternate: 1 1/2 cups of 1/2\&1/2 (milk and cream base, non-alcoholic)
        \item alternate: 1 1/2 cups Cream
        \item 1 cup sugar
        \item 4 egg yolks
        \item pinch of salt
        \item 3 teaspoons vanilla extract
        \item 1/2 cup sliced almonds
        \item 1 cup sliced almonds
        \item 2 tablespoons suger
        \item pinch of salt
    \end{itemize}
\end{framed}
\begin{enumerate}
    \item 
        In a medium saucepan over medium heat, cook 1/2 cup almonds with 2 tablespoons of sugar and a pinch of salt until deeply golden and caramelized (approx. 10 minutes). Transfer to a plate and set aside.
    \item 
        In the same pot, toast 1 cup sliced almonds until deeply golden (approx. 5 minutes). Proceed with base recipe in the same pot. Let custart steep off the heat for 1 hour before straining. 
    \item 
        Mix in the sweetened carmalized almonds. Chill.

\end{enumerate}

\subsubsection{Pistachio Flavor}
\begin{framed}
    \begin{itemize}
        \item alternate: 4 cups whole milk
        \item alternate: 1 cups of 1/2\&1/2 (milk and cream base, non-alcoholic)
        \item alternate: 1 cups Cream
        \item 1 cup sugar
        \item 4 egg yolks
        \item pinch of salt
        \item 3 teaspoons vanilla extract
        \item 1 cup pistachio paste 
        \item 1/4 teaspoon almond extract
    \end{itemize}
\end{framed}
\begin{enumerate}
    \item 
        Make the base ice cream. Whisk pistachio paste and almond extract into warm straned base. Chill
\end{enumerate}

\subsubsection{Peanut Butter Flavor}
\begin{framed}
    \begin{itemize} 
        \item alternate: 4 cups whole milk
        \item alternate: 1 cups of 1/2\&1/2 (milk and cream base, non-alcoholic)
        \item alternate: 1 cups Cream
        \item 1 cup sugar
        \item 4 egg yolks
        \item pinch of salt
        \item 3 teaspoons vanilla extract
        \item 1 cup natural smooth peanut butter 
        \item 1/2 teaspoon vanilla extract
    \end{itemize}
\end{framed}
\begin{enumerate}
    \item 
        Make the base ice cream. Whicks peanut butter and 1/2 teaspoon vanilla extract into warm, strained base. Chill.
\end{enumerate}

\subsubsection{Coconut Flavor}
\begin{framed}
    \begin{itemize}
        \item alternate: 2 cups whole milk
        \item alternate: 1 cups of 1/2\&1/2 (milk and cream base, non-alcoholic)
        \item alternate: 1 cups Cream
        \item 1 cup sugar
        \item 4 egg yolks
        \item pinch of salt
        \item 3 teaspoons vanilla extract
        \item 1 cup coconut milk
        \item 1/2 cup sweetened shredded coconut 
        \item 1 cup shredded unsweetened coconut
    \end{itemize}
\end{framed}
\begin{enumerate}
    \item 
        In a medium sacuepan, toast sweetened shredded coconut until deeply golden, about 5 minutes. Tansfer to a plate and set aside.
    \item 
        In the same pot, toast shredded unsweetened coconut until deeply golden, approx. 5 minutes. Proceed with base recipe in the same pot. Let custart steep off the heat for 1 hour befor straining. 
    \item 
        Mix in the cooked sweetened shredded coconut. Chill.

\end{enumerate}
\newpage

% =====================================================
\vspace*{\fill}
\begin{center}
    \section{Drinks}
\end{center}
\vspace*{\fill}
\newpage

% -----------------------------------------------------
\subsection{German Mulled Wine}
\noindent\rule[0.5ex]{\linewidth}{1pt}

%Ingredients
\begin{framed}
    \begin{itemize}
        \item 2 medium lemons
        \item 2 medium oranges
        \item 10 whole cloves
        \item 5 cardamon pods
        \item 1 1/4 cups granulated sugar
        \item 1 1/4 cups water
        \item 2 (3-inch) cinnamon sticks
        \item 2 (750-milliliter) bottles of dry red wine, such as Cabernet Sauvignon or Beajolais Nouveau
        \item 1/2 cup brandy
        \item Cheesecloth
        \item Butcher's twine
    \end{itemize}
\end{framed}

%Instructions
\begin{enumerate}
    \item 
        Using a vegetable peeler, remove the zest from the lemons and oranges in wide strips, avoiding the white pith; place the zest in a large saucepan. Juice the lemons and oranges and add the juice to the pan. Place the cloves and cardamom in a small piece of cheesecloth, tie it tightly with butcher’s twine, and add the bundle to the saucepan.
    \item 
        Add the sugar, water, and cinnamon sticks, place the pan over high heat, and bring to a simmer, stirring to dissolve the sugar. Reduce the heat to low and continue to simmer, stirring occasionally, until the mixture is reduced by about one-third, about 20 minutes.
    \item 
        Add the red wine and brandy, stir to combine, and bring just to a simmer (don’t let it boil). Remove from the heat and remove and discard the spice bundle before serving.
\end{enumerate}
\newpage

\end{document}

