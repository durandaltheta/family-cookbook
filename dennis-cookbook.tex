\documentclass{article}
\usepackage{framed}
\usepackage{fancyhdr}
\usepackage{lipsum}% just to generate text for the example
\pagestyle{fancy}
\fancyhf{}
\fancyhead[C]{\leftmark}
\fancyhead[R]{\thepage}
\renewcommand{\headrulewidth}{0pt}

\title{Dennis Family Cookbook}
\date{}
\begin{document}
\maketitle
\tableofcontents
\newpage

% =====================================================
\vspace*{\fill}
\begin{center}
    \section{Breakfasts}
\end{center}
\vspace*{\fill}
\newpage 

% -----------------------------------------------------
\subsection{Cowboy Coffee Cake [Christmas]} 
\noindent\rule[0.5ex]{\linewidth}{1pt}

%Ingredients
\begin{framed}
    \begin{itemize}
        \item 1 tbsp vinegar 
        \item ~1 cup milk
        \item 2 1/2 cups flour
        \item 2 cups brown sugar 
        \item 1/2 teaspoon salt 
        \item 2/3 cups shortening
        \item 2 tsp baking powder
        \item 1/2 tsp baking soda 
        \item 1/2 tsp cinnamon  
        \item 1/2 tsp nutmeg 
        \item 2 well beaten Eggs
    \end{itemize}
\end{framed}

%Instructions
\begin{enumerate}
    \item 
        Create sour milk by putting vinegar in an empty 1 cup measuring cup. Fill remaining space in the cup with with milk.
    \item 
        Combine flour, brown sugar, salt, and shortening till crumbly. Reserve 1/2 cup of mixture to crumble over batter later.
    \item 
        Add baking power, baking soda, cinnamon and nutmeg remaining batter. Mix thoroughly.
    \item 
        Add sour milk and well beaten eggs. Mix well.
    \item 
        Line 2 8x8 square pans with wax paper. Pour batter into pans. Sprinkle with reserved crumbs.
    \item 
        Bake at 375 degrees for approx. 25 minutes.
\end{enumerate}
\newpage

% -----------------------------------------------------
\subsection{Favorite Pancakes} 
\noindent\rule[0.5ex]{\linewidth}{1pt}

%Ingredients
\begin{framed}
\begin{itemize}
    \item 2 cups flour (500 ml)
    \item 2 tablespoons sugar, optional (30 ml)
    \item 4 teaspoons baking powder (20 ml)
    \item 1 teapsoon salt, optional (5 ml)
    \item 2 eggs, beaten
    \item 1 1/2 cups milk (370 ml)
    \item 1/4 cup oil or melted shortening (60 ml)
\end{itemize}
\end{framed}

%Instructions
\begin{enumerate}
    \item 
        Combine in a medium bowl the flower, sugar, baking powder and salt.
    \item 
        Combine seperately and then add to the main bowl the eggs and milk oil/shortening.
    \item 
        Stir quickly until blended. Do not beat. Cook on a hot, greased griddle, turning when bubly. Yields about 15 3-inch pancakes (8 cm)
\end{enumerate}

\paragraph
(Note: use any mixture of flours: whole wheat, white, oatmeal, rye, wheat germ, cornmeal, rice flour, millet, etc.)

% -----------------------------------------------------
\subsubsection{Apple Spice Pancakes} 
\noindent\rule[0.5ex]{\linewidth}{0.5pt}

%Ingredients
\begin{framed}
\begin{itemize}
    \item 1 cup grated apple (250 ml)
    \item 1 tablespoon lemon juice (15 ml)
    \item 2 tablespoons sugar (30 ml) 
    \item 1/2 teaspoon cinnamon (2 ml)
\end{itemize}
\end{framed}

%Instructions
\begin{enumerate}
    \item 
        Add the apple, lemon juice, sugar, and cinnamon to base mixture before cooking on griddle.
\end{enumerate}

% -----------------------------------------------------
\subsubsection{Banana Pancakes} 
\noindent\rule[0.5ex]{\linewidth}{0.5pt}

%Ingredients
\begin{framed}
\begin{itemize}
    \item 3/4 to 1 cup liquid mashed ripe bananas (180-250 ml)
    \item 1 tablespoon lemon juice (15 ml) 
    \item 2 tablespoons sugar (30 ml)
\end{itemize}
\end{framed}

%Instructions
\begin{enumerate}
    \item 
        Add the bananas, lemon juice, and sugar on to base mixture before cooking on griddle.
\end{enumerate}

% -----------------------------------------------------
\subsubsection{Cheese and Bacon Pancakes} 
\noindent\rule[0.5ex]{\linewidth}{0.5pt}

%Ingredients
\begin{framed}
\begin{itemize}
    \item 1/2 cup grated cheese 
    \item 1/2 cup crisp crumbled bacon
\end{itemize}
\end{framed}

%Instructions
\begin{enumerate}
    \item 
        Add the cheese and bacon on to base mixture before cooking on griddle.
\end{enumerate}

% -----------------------------------------------------
\subsubsection{Ham Pancakes} 
\noindent\rule[0.5ex]{\linewidth}{0.5pt}

%Ingredients
\begin{framed}
\begin{itemize}
    \item 1/2 to 1 cup ground or chopped ham (120-250 ml)
\end{itemize}
\end{framed}

%Instructions
\begin{enumerate}
    \item 
        Add the ham on to base mixture before cooking on griddle.
\end{enumerate}

% -----------------------------------------------------
\subsubsection{Pineapple Pancakes} 
\noindent\rule[0.5ex]{\linewidth}{0.5pt}

%Ingredients
\begin{framed}
\begin{itemize}
    \item 1 cup pineapple
    \item 3/4 cup pineapple juice
    \item 3/4 cup powdered milk
\end{itemize}
\end{framed}

%Instructions
\begin{enumerate}
    \item 
        Use the pineapple juice and powedered milk *INSTEAD* of regular milk in the base mixture. Add the pineapple on to base mixture before cooking on griddle.
\end{enumerate}

% -----------------------------------------------------
\subsubsection{Raisin Pancakes} 
\noindent\rule[0.5ex]{\linewidth}{0.5pt}

%Ingredients
\begin{framed}
\begin{itemize}
    \item 1 cup raisins (250 ml). 
\end{itemize}
\end{framed}

%Instructions
\begin{enumerate}
    \item 
        Add the raisins on to base mixture before cooking on griddle. Serve with jam.
\end{enumerate}

% -----------------------------------------------------
\subsubsection{Waffles} 
\noindent\rule[0.5ex]{\linewidth}{0.5pt}

%Instructions
\begin{enumerate}
    \item 
        Separate egg yolks from whites before beating eggs. Keep yolks in milk mixture. Fold in stiffly beaten whites into completed batter before cooking on griddle.
\end{enumerate}
\newpage

% =====================================================
\vspace*{\fill}
\begin{center}
    \section{Soups}
\end{center}
\vspace*{\fill}
\newpage

% -----------------------------------------------------
\subsection{Chili con Carne} 

%Ingredients
\begin{framed}
    \begin{itemize}
        \item 1 lb ground beef (or turkey)
        \item 1 C chopped onion
        \item 3/4 C green pepper
        \item 1 clove garlic, minced
        \item 1 28 oz can tomatoes, cut up
        \item 2 16 oz cans dark red kidney beans, drained
        \item 1 16 oz can tomato sauce
        \item 3 oz (1/2 can) tomato paste
        \item 1 T chili powder
        \item 1 tsp dried basil, crushed
        \item 1/2 tsp salt
        \item 1/4 tsp pepper
    \end{itemize}
\end{framed}

%Instructions
\begin{enumerate}
    \item
        In a large kettle cook ground beef, onion, green pepper, and garlic until meat is browned. Drain off fat.
    \item
        Stir in undrained tomatoes, kidney beans, tomato sauce, chili powder, basil, salt, and pepper. Bring to boiling; reduce heat.
    \item
        Cover and simmer about 20 minutes
    \item
        Makes 4-6 servings
\end{enumerate}
\newpage

% -----------------------------------------------------
\subsection{Caldo Verde (Spicy Sausage and Kale Sopa)} 
\noindent\rule[0.5ex]{\linewidth}{1pt}

%Ingredients
\begin{framed}
    \begin{itemize}
        \item 1/4 cup olive oil
        \item 1 cup chopped onion
        \item 2 tsp chopped garlic
        \item 5 cups Ihaho potatoes, peeled and thinly sliced
        \item 1 quart water
        \item 1 quart chicken broth
        \item 6 oz chorizo sausage, thinly sliced
        \item salt and black pepper
        \item 1 lb kale, washed, trimmed of the thick stems and thinly sliced
    \end{itemize}
\end{framed}

%Instructions
\begin{enumerate}
    \item 
        In a medium soup pot, heat 3 tablespoons of olive oil, add onions and garlic and cook for 2 to 3 minutes until they turn glassy, don't let them get brown. 
    \item 
        Add potatoes and water. Cover and boil gently over medium heat for 20 minutes. Meanwhile, in a skillet cook sausage until most of the fat is rendered out. Drain and set aside. 
    \item 
        When the potatoes are tender, mash them with a potato masher right in the pot. Add sausage to the soup then add kale. Simmer for 5 minutes. Add the remaining olive oil and season. Ladle into bowls and serve.
\end{enumerate}
\newpage

% -----------------------------------------------------
\subsection{Fruit Soup [Christmas]} 
\noindent\rule[0.5ex]{\linewidth}{1pt}

%Ingredients
\begin{framed}
    \begin{itemize}
        \item 1 cup raisins 
        \item 1 cup prunes 
        \item 1/2 cup rice or tapioca 
        \item 1 orange 
        \item 1 lemon 
        \item 1 cup sugar 
        \item 2 cinnamon sticks 
        \item 2 apples
        \item Any fruit juice cocktail 
        \item 2 quarts water 
    \end{itemize}
\end{framed}

%Instructions
\begin{enumerate}
    \item 
        Peel and dice apples. Juice orange and lemon.
    \item 
        Place all ingredients in a large pot. Simmer slowly for several hours stirring occasionally until thickened.
\end{enumerate}
\newpage

% =====================================================
\vspace*{\fill}
\begin{center}
    \section{Breads}
\end{center}
\vspace*{\fill}
\newpage

% -----------------------------------------------------
\subsection{Pioneer Cornbread} 
\noindent\rule[0.5ex]{\linewidth}{1pt}

%Ingredients
\begin{framed}
    \begin{itemize}
        \item 1 1/3 C Cold Margerine or Butter (frozen is best)
        \item 1 C sugar
        \item 2 Eggs
        \item 1 C cornmeal
        \item 1 tsp salt
        \item 2 C flour
        \item 1 T baking powder
        \item 2 C milk
    \end{itemize}
\end{framed}


%Instructions
\begin{enumerate}
    \item 
        Cut margerine/butter into small cubes.
    \item 
        Mix margerine/butter with sugar and eggs. Mix well and make sure the cubes are well separated and not clumped together
    \item 
        Add remaining ingredients and stir until moistened
    \item 
        Pour into a greased 9x13 pan or two greased cast iron skillets
    \item 
        Bake at 350 degrees for 20-30 min. Cornbread is done when a toothpick inserted in the middle comes out clean
    \item 
        If desired: remove pan from the oven and turn on the broiler, move the rack to the top position and brown the cornbread under the boiler for 30 seconds or so.  Watch carefully the entire time.  The bread can turn from brown to burned in seconds!
\end{enumerate}
\newpage

% =====================================================
\vspace*{\fill}
\begin{center}
    \section{Appetizers \& Side Dishes}
\end{center}
\vspace*{\fill}
\newpage

% -----------------------------------------------------
\subsection{Cranberry Cherry Jello Salad} 
\noindent\rule[0.5ex]{\linewidth}{1pt}

%Ingredients
\begin{framed}
    \begin{itemize}
        \item 2 cups fresh cranberrys
        \item 1 cup sugar
        \item 1 1/2 cup water
        \item 3 oz cherry jello 
        \item 3/4 cup chopped celery 
        \item 1/2 cup chopped nuts 
        \item 1/2 cup chopped apples
    \end{itemize}
\end{framed}

%Instructions
\begin{enumerate}
    \item 
        Boil berries, sugar, and water until berries pop, remove from heat.
    \item 
        Add jello and let cool slightly.
    \item 
        Mix in celery, nuts and apples. Put in fridge to chill for a couple hours.
\end{enumerate}
\newpage

% -----------------------------------------------------
\subsection{Hummus} 
\noindent\rule[0.5ex]{\linewidth}{1pt}

%Ingredients
\begin{framed}
    \begin{itemize}
        \item 15 ounce can (425 grams) chickpeas, also called garbanzo beans
        \item 1/4 cup (59 ml) fresh lemon juice, about 1 large lemon
        \item 1/4 cup (59 ml) well-stirred tahini
        \item Half of alarge garlic clove, minced
        \item 2 tablespoons olive oil, plus more for serving
        \item 1/2 to 1 teaspoon kosher salt, depending on taste
        \item 1/2 teaspoon ground cumin
        \item 2 to 3 tablespoons water
        \item Dash of ground paprika for serving
    \end{itemize}
\end{framed}

%Instructions
\begin{enumerate}
    \item 
        In the bowl of a food processor, combine tahini an dlemon juice. Process for 1 minute Scrape sides and bottom of bowl then turn on and process for 30 seconds. This extra time helps "whip" or "cream" the tahini, making smooth and creamy hummus possible.
    \item 
        Add te olive oil, minced garlic, cumin and the salt to the whipped tahini and lemon juice mixture. Process for 30 seconds, scrape sides and bottom of bowl then process another 30 seconds.
    \item 
        Open can of chicpeaks, drain liquid then rinse well with water. Add half of the chickpeas to the food processor then process for 1 minute. Scrape sides and bottom of bowl, add remaining chickpeas and process for 1 to 2 minutes or until thick and quite smooth.
    \item 
        Most likely the hummus will be too thick or still have tiny bits of chickpea. To fix this, with the food processor turned on, slowly add 2 to 3 tablespoons of water until the consistency is perfect.
    \item 
        Scrape the hummus into a bowl, then drizzle about 1 tablespoon of olive oil over the top and sprinkle the paprika.
\end{enumerate}

\paragraph 
(Note: Store homemade hummus in an airtight container and refigerate up to one week.)
\newpage

% -----------------------------------------------------
\subsection{Red Hot Jello} 
\noindent\rule[0.5ex]{\linewidth}{1pt}

%Ingredients
\begin{framed}
    \begin{itemize}
        \item 1/2 cup water
        \item 1 pkg red jello 
        \item 1/4 cup red hot candies (generally cinnamon candies)
        \item 1 applesauce
    \end{itemize}
\end{framed}

%Instructions
\begin{enumerate}
    \item 
        Boile water and add water to red jello and candies. Dissolve candies.
    \item 
        Mix in applesauce. Chill in fridge for a couple hours.
\end{enumerate}
\newpage

% =====================================================
\vspace*{\fill}
\begin{center}
    \section{Vegetables}
\end{center}
\vspace*{\fill}
\newpage

% =====================================================
\vspace*{\fill}
\begin{center}
    \section{Entrees}
\end{center}
\vspace*{\fill}
\newpage

% -----------------------------------------------------
\subsection{Gourmet Four Cheese Macaroni and Cheese} 
\noindent\rule[0.5ex]{\linewidth}{1pt}

%Ingredients
\begin{framed}
    \begin{itemize}
        \item 1 lb rotelle pasta (they look like wagon wheels)
        \item 3/4 lb sharp cheddar cheese, shredded
        \item 1/2 lb gruyer cheese, shredded
        \item 1/2 cup asiago cheese, shredded
        \item 1/2 cup Fontina cheese, shredded
        \item 3 tablespoons unsalted butter
        \item 3 tablespoons all-purpose flour
        \item 2 cups milk (2\%)
        \item 1 teaspoon onion powder
        \item 1/2 teaspoon salt
        \item 1/4 teaspoon dried mustard
        \item 1/4 teaspoon nutmeg
        \item 1/4 teaspoon ground cayenne pepper
        \item 1 cup panko breadcumbs (japanese bread crumbs)
    \end{itemize}
\end{framed}

%Note
\paragraph 
(This is a BAKED macaroni and cheese, therefore, it will NOT turn out with a lot of extra cheesy sauce, as it is absorbed into the pasta while baking. If you requre and extra saucy mac \& cheese, just reduce the amount of pasta you place into the dish or make more sauce so it is creamier.)

%Instructions
\begin{enumerate}
    \item 
        Heat oven to 350 degrees. Coat a 3 quart rectangular baking dish with non-stick spray. Bring a large pot of lightly salted water to boiling.
    \item 
        Toss all the shredded cheeses together in a large bowl, set aside.
    \item 
        Melt butter in a medium-sized saucepan over medium heat. Whisk in the flour until smooth and slightly bubbly.
    \item 
        In a thin stream, whisk in the milk. Stir in the onion powder, salt, nutmeg, dried mustard and cayenne.
    \item 
        Bring to a boil over medium high heat. Reduce heat and simmer 3 minutes. Remove from heat; whisk in a 2 1/2 cups of the cheese mixture and stir until smooth. Cover to retain heat.
    \item  
        Once the water boils, add pasta. Cook until your desired doneness, then drain. In the pasta container stir together the cooked pasta and cheese sauce.
    \item 
        Pour half of the mixture into the prepared dish. Sprinkle with a generous cup of the reserved cheeese. Spoon remaining chees-covered pasta into the dish and top with the remaining cheese.
    \item 
        Add 1 cup of japanese panko bread crumbs to the top of the mixture.
    \item 
        Bake at 350 degrees for 30 minutes or until the pank curmbls are lightly browned and the cheese is bubbly. Cool slightly before serving.
\end{enumerate}
\newpage

% -----------------------------------------------------
\subsection{Red, White, and Blue Mac and Cheese} 
\noindent\rule[0.5ex]{\linewidth}{1pt}

%Ingredients
\begin{framed}
    \begin{itemize}
        \item 3 tablespoons butter
        \item 1/4 cup flour
        \item 1 1/2 cups half and half
        \item 1 1/2 cups heavy cream
        \item 2 cloves garlic, finely grated
        \item 1/4 teaspoon cayenne pepper
        \item 1 teaspoon Worcestershire sauce
        \item Salt an dblack pper to taste
        \item 1 cup sun-dried tomatoes, drained and sliced
        \item 1 3/4 cups (7 ounces) Wisconsi white chedder cheese, shredded and divided
        \item 1 3/4 cups (7 ounces) Wisconsin montery jack cheese, shredded and divided
        \item 1 1/2 cups (9 ounces) Wisconsin blue cheese, crumbled and dividied
        \item 1 pound cavatappi pasta cooked to al dente, drained and cooled
        \item paprika, optional
    \end{itemize}
\end{framed}

%Instructions
\begin{enumerate}
    \item 
        Preheat oven to 350 degrees.
    \item 
        In a medium pot over medium heat, melt butter. Add flower and whisk for 3 minutes until lightly browned.
    \item 
        Add heavy whipping cream, half and half, garlic, cayenne, Worcestershire sauce, salt and pepper. Stir for about 3 minutes over medium heat until slightly thick.
    \item 
        Remove from heat and add sun dried tomatoes and cheeses, reserving 1/2 cup whit chedder, 1/2 cup monterey jack, and 1/2 cup blue cheese. Stir until chees is melted into sauce.
    \item 
        Butter 13x9 baking dish. In lare pot, add sauce to pasta. Stir until pasta is evenly coated. Pour into baking dish and spread evenly. Sprinkle reserved cheese over top. Sprinkle with paprika if desired.
    \item 
        Bake 25-30 miutes until bubbly. Le tsit 5-10 minutes before serving.
\end{enumerate}
\newpage

% =====================================================
\vspace*{\fill}
\begin{center}
    \section{Deserts}
\end{center}
\vspace*{\fill}
\newpage

% -----------------------------------------------------
\subsection{Banana Cupcakes with Honey Cinnamon Frosting} 
\noindent\rule[0.5ex]{\linewidth}{1pt}

%Ingredients
\begin{framed}
    \begin{itemize}
        \item 1 1/2 cups all-purpose flour, (spooned and leveled)
        \item 3/4 cup sugar
        \item 1 teaspoon baking powder
        \item 1/2 teaspoon baking soda
        \item 1/4 teaspoon salt
        \item 1/2 (1 stick) unsalted butter, mleted
        \item 1 1/2 cups mashe bananas (about 4 ripe bananas), plus 1 whole banana, for garnish (optional)
        \item 2 large eggs
        \item 1/2 teaspoon pure vanilla extract
        \item Honey-Cinnamon Frosting
    \end{itemize}
\end{framed}

%Instructions
\begin{enumerate}
    \item 
        Preheat oven to 350 degrees. Line a standard 12-cup muffin pan with paper liners. In a medium bowl, whisk together flour, sugar, baking powder, baking soda, and salt.
    \item 
        Make a well in center for flour mixture. In well, mix together butter, mashed bananas, eggs, and vanilla. Stir to incorporate flour mixture (do not overmix). Dividing evenly, spoon batter into muffin cups.
    \item 
        Bake until a toothpick inserted in center of a cupcake comes out clean, 25 to 30 minutes. Remove cupcakes from pan: cool completely on a wire rack. Spread tops with Honey-Cinnamon Frosting.
    \item 
    \item 
\end{enumerate}

\subsubsection{Honey Cinnamon Frosting}
\noindent\rule[0.5ex]{\linewidth}{0.5pt}

%Ingredients
\begin{framed}
    \begin{itemize}
        \item 1 1/4 cup confectioner's sugar 
        \item 1/2 cup (1 stick) unsalted butter, room temperature
        \item 1 tablespoon honey
        \item 1/8 teaspoon ground cinnamon
    \end{itemize}
\end{framed}

%Instructions
\begin{enumerate}
    \item 
        In a medium bowl, using an electric mixer, beat confectioners' sugar, unsalted butter, honey, and ground cinnamon until smooth, 4 to 5 minutes
\end{enumerate}
\newpage
\newpage

% -----------------------------------------------------
\subsection{Crock Pot Candy} 
\noindent\rule[0.5ex]{\linewidth}{1pt}

%Ingredients
\begin{framed}
    \begin{itemize}
        \item 32 oz. mixed nuts with extra cashews 
        \item 12 oz. semi sweet chocolate chips 
        \item 1 bakers 4 ozs. german chocolate bar
        \item 32 oz white chocolate chips
    \end{itemize}
\end{framed}

%Instructions
\begin{enumerate}
    \item 
        Put mixed nuts in crockpot. 
    \item 
        Pour chocolate chips over nuts.
    \item 
        Break german chocolate bar over top.
    \item 
        Pour white chocolate chips over everything.
    \item 
        Cook on low (DO NOT REMOVE LID) for 2 hours.
    \item 
        Stir well and drop on waxed paper to cool.
\end{enumerate}
\newpage

% -----------------------------------------------------
\subsection{Magnolia's Chocolate Cupcakes} 
\noindent\rule[0.5ex]{\linewidth}{1pt}

%Ingredients
\begin{framed}
    \begin{itemize}
        \item 2 cups all purpose flower
        \item 1 teaspoon baking soda
        \item 1 cup (2 sticks) unsalted butter, softened
        \item 1 cup granulated suger
        \item 1 cup firmly packed light brown suger
        \item 4 large eggs, at room temperature
        \item 6 ounces unsweetened chocolate, melted (see note)
        \item 1 cup buttermilk
        \item 1 teaspoons vanilla extract
        \item Vanilla Buttercream (recipe follows) or Chocolate Buttercream
    \end{itemize}
\end{framed}

%Instructions
\begin{enumerate}
    \item  
        Preheat oven to 350 degrees.
    \item 
        Line two 12-cup muffin tins with cupcake papers. Set aside
    \item  
        In a small bowl, sift together the flour and baking soda. Set asid.
    \item  
        In a large bowl, on the medium speed of an electric mixer, cream the butter until smooth. Add the sugars and beat until fluffy, about 3 minutes. Add the eggs, one at a time, beating well after each addition. Add the chocolate, mixing until well incorporated
    \item 
        Add the dry ingredients in three parts, alternating with the buttermilk and vanilla. With each addition, beat until the ingredients are incorportaed, but do not overbeat. 
    \item 
        Using a rubber spatula scrape down the batter in the bowl to make sure the ingredients are well blended and the batter is smooth. Caerfully spoon the batter into the cupcake liners, filling them about three-quarters full. 
    \item 
        Bake for 20-25 minutes, or until a caker tester inserted inthe center of the cupcake comes out clean.
    \item 
        Cool the cupcakes in th etins for 15 minutes. Remove from the tins and cool completely on a wire rack before icing. 
    \item 
        Ice the cupcakes either with Vanilla Buttercream or Chocolate Buttercream
\end{enumerate}

\paragraph 
(Note: If you would like to make a layer cake instead of cupcakes divide the batter between two 9-inch round cake pans and bake the layers for 30-40 minutes)

\subsubsection{Magnolia's Vanilla Buttercream Frosting}
\noindent\rule[0.5ex]{\linewidth}{0.5pt}

%Ingredients
\begin{framed}
    \begin{itemize}
        \item 1 cup (2 sticks) unsalted butter, softed
        \item 6 to 8 cups confectioners' sugar
        \item 1/2 cup milk
        \item 2 teaspoons vanilla extract
    \end{itemize}
\end{framed}

%Instructions
\begin{enumerate}
    \item 
        Place the butter in a large mixing bowl. Add 4 cups of the sugar and then the milk and vanilla. On the medium speed of an electric mixer, beat until smooth and creamy, about 3-5 minutes.
    \item 
        Gradually add the remaining sugar, 1 cup at a time, beating well after each addition (about 2 minutes), until the icing is thick enough to be of good spreading consistency. You may not need to add all of the sugar.
    \item 
        If desired, add a few drops of food coloring and mix thoroughly. (Use and store the icing at room temperature because icing will set if chilled.) Icing can be stored in an airtight container for up to 3 days
\end{enumerate}

\paragraph
(Note: if you are icing a 3-layer cake, use the folloing recipe proportions:
\begin{itemize}
    \item 1 1/2 cups (3 sticks) unsalted butter
    \item 8 to 10 cups confectioners' sugar
    \item 3/4 cup milk
    \item 1 tablespoon vanilla extract
\end{itemize}
)

\subsubsection{Magnolia's Chocolate Buttercream Frosting}
\noindent\rule[0.5ex]{\linewidth}{0.5pt}

%Ingredients
\begin{framed}
    \begin{itemize}
        \item 1 1/2 cup unsalted butter
        \item 2 tablespoons mil
        \item 9 ounces, weight semisweet chocolate, melted and cooled to lukewarm (see note)
        \item 1 teaspoon Vanilla Extract
        \item 1 1/4 cups powdered sugar, sifted
    \end{itemize}
\end{framed}

%Instructions
\begin{enumerate}
    \item 
        In a large mixing bowl, beat the butter using an electric mixer on a medium speed for about 3 minutes or until creamy.
    \item 
        Add the milk carefully and beat until smooth. 
    \item 
        Add the melted chocolate and beat well for 2 minutes. 
    \item 
        Add the vanilla and beat for 3 minutes
    \item 
        Gradually add in the sugar and beat on low speed until creamy and of desired consistency. (Yields enough frosting for a 9-inch 2-layer cake or about 2 dozen cupcakes.
\end{enumerate}

\paragraph
(Note: to melt the chocolate, place it in a double boiler over simmering water on low heat for 5-10 minutes. Stir occasionally until the chocolate is completely smooth and no pieces remain. Remove from heat and let cool 5-15 minutes or until lukewarm.)
\newpage

% -----------------------------------------------------
\subsection{Mystery Desert} 
\noindent\rule[0.5ex]{\linewidth}{1pt}

%Ingredients
\begin{framed}
    \begin{itemize}
        \item 2 cup flower 
        \item 1 cup Butter
        \item 1 cup chopped walnuts
        \item 9 oz cool whip
        \item 8 oz cream cream cheese 
        \item 2/3 cup powdered sugar
        \item 2 packages chocolate instant Pudding
        \item cool whip (as a topping)
    \end{itemize}
\end{framed}

%Instructions
\begin{enumerate}
    \item 
        Bake flour, butter, and walnuts at 350 degrees 15minutes on 9x13 inch pan. Let cool
    \item 
        Mix 9 oz cool whip, cream cheese and powdered sugar together. Layer mixtur on top of bottom crumble.
    \item 
        Make chocolate pudding according to instructions on package. Layer pudding on top of previous layer in pan.
    \item 
        Layer cool whip as final layer on pan. 
\end{enumerate}
\newpage

% -----------------------------------------------------
\subsection{Rice Pudding [Christmas]} 
\noindent\rule[0.5ex]{\linewidth}{1pt}

%Ingredients
\begin{framed}
    \begin{itemize}
        \item 1 cup white rice.
        \item approx. 2 cups water
        \item 1 quart milk 
        \item 1 can evaporated milk 
        \item 1 cup sugar 
        \item 3 beaten eggs 
        \item 1 tsp vanilla extract
        \item Raisins to taste
        \item pinch of cinnamon
    \end{itemize}
\end{framed}

%Instructions
\begin{enumerate}
    \item 
        Cook rice according to packaging directions.
    \item 
        Add milk and evaporated milk to rice in a pot. Bring to a boil stirring often. Add sugar, stir, and boil *slowly* for about 20 minutes stirring constantly. 
    \item 
        Beat eggs in a separate bowl and slowly add a tbsp or two of the rice mixture at a time to the eggs, stirring well after each addition, until you've added about 1 cup of rice mixture to the eggs and both are mixed together well. 
    \item 
        Mix egg/rice mixture into the original pot with the rest of the heated rice mixture. Boil an additional 4 or 5 minutes stirring slowly.
    \item 
        Remove form heat, add vanilla extract and raisins. Mix well. Sprinkle cinnamon on top. Chill for a few hours till quite cool.
\end{enumerate}
\newpage

% -----------------------------------------------------
\subsection{Tom Dennis Master Custard Ice Cream Recipe} 
\noindent\rule[0.5ex]{\linewidth}{1pt}

%Ingredients
\begin{framed}
    \begin{itemize}
        \item 2 cups whole milk
        \item 1 cup sugar
        \item 4 egg yolks
        \item pinch of salt
        \item 2 cups of 1/2\&1/2 (milk and cream base, non-alcoholic)
        \item 2 cups Cream
        \item 3 teaspoons vanilla extract
    \end{itemize}
\end{framed}

%Instructions
\begin{enumerate}
    \item 
        In pan wisk milk, sugar, egg yokes, and salt on medium heat until mixture simmers.
    \item 
        Lower heat, wisk 5 minutes till mixture thickens.
    \item 
        Strain into a bowl and wisk in 1/2\&1/2, cream, and vanilla.
    \item 
        Churn in an ice cream machine according to manufacturer's instructions. Serve directly from machine for soft serve, or store in freezer till needed
\end{enumerate}

%Flavors
\noindent\rule[0.5ex]{\linewidth}{0.5pt}
\paragraph 
(The following flavors can create thicker custard than usual, be aware that manual churning may be required if the ice cream machine is not powerful enough. Alternatively, one can flip the ratio of milk to cream base to the following alternate measurements to thin the recipe as provided in each flavor.)

\subsubsection{Almond Flavor}
\noindent\rule[0.5ex]{\linewidth}{0.5pt}
\begin{framed}
    \begin{itemize}
        \item alternate: 3 cups whole milk
        \item alternate: 1 1/2 cups of 1/2\&1/2 (milk and cream base, non-alcoholic)
        \item alternate: 1 1/2 cups Cream
        \item 1 cup sugar
        \item 4 egg yolks
        \item pinch of salt
        \item 3 teaspoons vanilla extract
        \item 1/2 cup sliced almonds
        \item 1 cup sliced almonds
        \item 2 tablespoons suger
        \item pinch of salt
    \end{itemize}
\end{framed}
\begin{enumerate}
    \item 
        In a medium saucepan over medium heat, cook 1/2 cup almonds with 2 tablespoons of sugar and a pinch of salt until deeply golden and caramelized (approx. 10 minutes). Transfer to a plate and set aside.
    \item 
        In the same pot, toast 1 cup sliced almonds until deeply golden (approx. 5 minutes). Proceed with base recipe in the same pot. Let custart steep off the heat for 1 hour before straining. 
    \item 
        Mix in the sweetened carmalized almonds. Chill.

\end{enumerate}

\subsubsection{Pistachio Flavor}
\noindent\rule[0.5ex]{\linewidth}{0.5pt}
\begin{framed}
    \begin{itemize}
        \item alternate: 4 cups whole milk
        \item alternate: 1 cups of 1/2\&1/2 (milk and cream base, non-alcoholic)
        \item alternate: 1 cups Cream
        \item 1 cup sugar
        \item 4 egg yolks
        \item pinch of salt
        \item 3 teaspoons vanilla extract
        \item 1 cup pistachio paste 
        \item 1/4 teaspoon almond extract
    \end{itemize}
\end{framed}
\begin{enumerate}
    \item 
        Make the base ice cream. Whisk pistachio paste and almond extract into warm straned base. Chill
\end{enumerate}

\subsubsection{Peanut Butter Flavor}
\noindent\rule[0.5ex]{\linewidth}{0.5pt}
\begin{framed}
    \begin{itemize} 
        \item alternate: 4 cups whole milk
        \item alternate: 1 cups of 1/2\&1/2 (milk and cream base, non-alcoholic)
        \item alternate: 1 cups Cream
        \item 1 cup sugar
        \item 4 egg yolks
        \item pinch of salt
        \item 3 teaspoons vanilla extract
        \item 1 cup natural smooth peanut butter 
        \item 1/2 teaspoon vanilla extract
    \end{itemize}
\end{framed}
\begin{enumerate}
    \item 
        Make the base ice cream. Whicks peanut butter and 1/2 teaspoon vanilla extract into warm, strained base. Chill.
\end{enumerate}

\subsubsection{Coconut Flavor}
\noindent\rule[0.5ex]{\linewidth}{0.5pt}
\begin{framed}
    \begin{itemize}
        \item alternate: 2 cups whole milk
        \item alternate: 1 cups of 1/2\&1/2 (milk and cream base, non-alcoholic)
        \item alternate: 1 cups Cream
        \item 1 cup sugar
        \item 4 egg yolks
        \item pinch of salt
        \item 3 teaspoons vanilla extract
        \item 1 cup coconut milk
        \item 1/2 cup sweetened shredded coconut 
        \item 1 cup shredded unsweetened coconut
    \end{itemize}
\end{framed}
\begin{enumerate}
    \item 
        In a medium sacuepan, toast sweetened shredded coconut until deeply golden, about 5 minutes. Tansfer to a plate and set aside.
    \item 
        In the same pot, toast shredded unsweetened coconut until deeply golden, approx. 5 minutes. Proceed with base recipe in the same pot. Let custart steep off the heat for 1 hour befor straining. 
    \item 
        Mix in the cooked sweetened shredded coconut. Chill.

\end{enumerate}
\newpage

% =====================================================
\vspace*{\fill}
\begin{center}
    \section{Drinks}
\end{center}
\vspace*{\fill}
\newpage

% -----------------------------------------------------
\subsection{German Mulled Wine [Christmas]}
\noindent\rule[0.5ex]{\linewidth}{1pt}

%Ingredients
\begin{framed}
    \begin{itemize}
        \item 2 medium lemons
        \item 2 medium oranges
        \item 10 whole cloves
        \item 5 cardamon pods
        \item 1 1/4 cups granulated sugar
        \item 1 1/4 cups water
        \item 2 (3-inch) cinnamon sticks
        \item 2 (750-milliliter) bottles of dry red wine, such as Cabernet Sauvignon or Beajolais Nouveau
        \item 1/2 cup brandy
        \item Cheesecloth
        \item Butcher's twine
    \end{itemize}
\end{framed}

%Instructions
\begin{enumerate}
    \item 
        Using a vegetable peeler, remove the zest from the lemons and oranges in wide strips, avoiding the white pith; place the zest in a large saucepan. Juice the lemons and oranges and add the juice to the pan. Place the cloves and cardamom in a small piece of cheesecloth, tie it tightly with butcher’s twine, and add the bundle to the saucepan.
    \item 
        Add the sugar, water, and cinnamon sticks, place the pan over high heat, and bring to a simmer, stirring to dissolve the sugar. Reduce the heat to low and continue to simmer, stirring occasionally, until the mixture is reduced by about one-third, about 20 minutes.
    \item 
        Add the red wine and brandy, stir to combine, and bring just to a simmer (don’t let it boil). Remove from the heat and remove and discard the spice bundle before serving.
\end{enumerate}
\newpage

\end{document}

